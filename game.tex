\documentclass{article}
\usepackage{graphicx}
\begin{document}
\title{A game of little men}

\begin{titlepage}
\begin{huge}
Mechanics Documentation
\end{huge}
\end{titlepage}

\pagebreak

\tableofcontents

\pagebreak

\section{Basic Playing Components}
\subsection{Playing Materials}
\paragraph{•}
You will need some basic materials and items in order to play the game. These are:
\begin{itemize}
\item Some d6 dice
\item Some models based as will be indicated or proxies
\item Some measuring device like a ruler or tape measure
\item A flat surface to play on
\end{itemize}

This game is designed to be highly scalable so players are welcome to play with as simple or as complex materials as they wish.

The use of terrain pieces will no doubt add greater complexity and enjoyment to the game but the game is also playable without it as more than not, battles were fought in plainly open fields.

\section{Unit Components and Characteristics}
\subsection{The Unit}
The basic component of the battle system described here is the unit. Depending on the scale being considered, by the players, this could be anything from half a dozen men, to a collection of one hundred or more. As long as both players are in agreement to the scale of the conflict and the basing requirements, the unit is a simple affair.

\subsection{Units and Orientation}
Units have an orientation, that will determine what they are facing at currently, and therefore currently able to see. This is defined in terms of 4 facings or orientations: Front,Rear, Left Flank and Right Flank. A unit whose orientation is indicated in terms of the way the models are positioned, in terms of them all generally facing the same direction, will general face forward and perform actions in terms of a forward aligned sense.

The other orientations are important as they would be relative "blind spots" for the unit in terms of where it would be able to see and where it would not. After all, a unit that finds an enemy in its rear or flanks is going to have a bad time in close combat.

Please refer to the diagram below for illustration of the facings. Note that this applies in all cases except otherwise where noted.

\begin{figure}
\includegraphics[scale=0.55]{Unit}
\caption{A diagram showing a unit}
\end{figure}

In the figure above, the unit is made up of 4 stands. Each stand as a number of models placed on it, as indicated by the circles, and as a whole, the unit is facing in the direction of the arrow. The unit's other orientations are indicated as shown.

\subsection{The Unit and Stands}
In truth, a unit is represented by placing a number of flat rectangles, onto which models are typically placed, called bases or stands into touching contact, or base-to-base, contact. In this way, a unit is made up of several stands. These will then give us a quick and easy way of immediately identifying the relative strength of a unit since the number of stands in the unit will indicate the number of attacks, potential damage it can take and so on. Unless otherwise stated, stands of a unit must always remain in base contact with each other except when removing casualties.

In general, a unit will occupy a space on the table, a flat rectangular presence called a base, that has the following dimensions: 

\paragraph{•}
The unit stand sizes are listed below:
\begin{itemize}
\item 50mm by 50mm for artillery
\item 40mm by 40mm for all non-artillery units
\end{itemize} 

\subsection{Stand Positioning and Formation}
\paragraph{•}
The way stands are positioned in a unit a simple one in most circumstances. Generally speaking, stands will be in base-to-base contact with each other and stay in this way as the unit moves on the battlefield. However, there are a number of instances where this may change, either in accordance with your wishes or due to extraneous circumstances.

\paragraph{•}
The first of these is a Formation Change. Many times, you will want to change the positions of your stands such that more stands may shoot their weapons or perhaps your unit is better formed up for melee combat. It is even the case that you may need to adopt a different stand structure to move past certain terrain. When this happens, we say that your unit's formation is changing, that is you are performing a Formation Change.

A Formation Change is a type of order that will later be discussed in the Orders section of this document. Now we will focus on the valid ways that a formation, as comprised of stands of a unit, can be restructured in terms of expressing the requirements for the validity of a formation.

\paragraph{•}
A formation is valid if it conforms to the following:
\begin{itemize}
\item No stand does not touch at least one other stand in the unit
\item When grouping units into columns or rows, files or ranks, the ranks must be consistently sized the same as far as possible, starting from the front of the unit, before allowing for incomplete ranks or files. For example, if you want 3 stands across for a rank, and 7 units, you will get a final unit with 2 ranks of 3 stands and one incomplete rank of 1 stand at the very back. How many ranks or files you wish a unit to have is up to you but remember the number of stands available
\end{itemize}

Certain configurations of stands in a unit, for different units, will produce different effects. Generally speaking, a unit that is longer, rank wise, than column wise is harder to attack in melee whereas a unit with more files than ranks generally tends to be able to shoot more and suffer less from artillery.

\paragraph{•}
The other reason to have to change formation of unit stands is that there have been casualties in the unit in terms of lost stands and suddenly the unit is no longer legally structured. When applicable, you must ensure that you reform the unit such that it is legally structured.
\subsection{Unit Types}
The unit will be one of many types:
\begin{itemize}
\item Pikemen: Men armed with long wooden poles tipped with metal. Pikemen are very useful to fend off cavalry and protect Musketeers from being engaged in melee while they load their weapons. Pikemen, however, are rarely well armoured and will be required to take much enemy fire.
\item Swordsmen: Swordsmen are specially trained soldiers armed with swords and bucklers, or rarely shields, who are lightly armoured and instructed to support other soldiers in close quarters combat, such as pikemen, where their more agile swords are at a distinct advantage over the unwieldy pikes.  
\item Heavy Infantry: Heavy infantry are men armed with more singly powerful weapons like two-handed swords or halberds and wearing much more extensive, and expensive, armour who are directed to either directly assault enemy units or absorb the charge on behalf of other more vulnerable units. They are too expensive to typically deploy en masse but they find great use in sieges.
\item Musketeers: Men armed with primitive but deadly blackpowder weapons. So armed, these men are able to fire lead balls at great distance, albeit at a slow rate. This necessitates the use of Pikemen to protect them from enemy infantry and cavalry who would otherwise kill them.
\item Crossbowmen: The crossbow is a traditional weapon, with a usage ranging well before the advent of gunpowder. Although the sun is certainly setting on the time of the crossbow, certainly there are still many soldiers, and cities, that rely upon the humble crossbow for defence against enemy attack.
\item Cavalry: Men armed with various manner of hand weapons ranging from sabres to lances, mounted upon horses. Although usually lightly armoured, cavalry are deadly in melee owing to their speed and tenacity.
\item Dragoons: A dragoon is a specialist cavalry trooper that is armed with a small firearm called a carbine that allows them to fire and reload whilst on horseback. This comes at a great cost but allows soldiers great flexibility to move around the battlefield and bring firepower where it is needed.
\item Artillery: Artillery units are composed of field guns, smaller pieces that are more maneuverable than larger siege weapons. These artillery units are inaccurate and vulnerable but nevertheless very destructive weapons of war.
\item Standard Bearers: This refers to a collection of soldiers who are specifically entrusted to carry battle standards, as well as musicians, who are entrusted to carry the honour of the army on the field and bolster the resolve of the soldiers.
\item Novice: Novices are commanders that are either extremely new to the practice of leading soldiers in warfare, owing perhaps to an unexpected field promotion some battles prior or are graduates of a formal academy that have yet to see proper battle. Well meaning certainly, and not without their own level of skill, Novices are plentiful commanders that should not be depended upon too heavily in a battle. Their relative commonness makes them ideal for leading regiments with highly dangerous tasks like storming breaches or those guarding artillery.
\item Hero: Heroes are somewhat veteran commanders who have shown leadership in battle. They may be lower-ranked nobles, or particularly seasoned commoners who have spent a lifetime soldiering. Whatever their story, they are the most common of leaders on the field and will typically provide the bulk of the instruction
\item General: Generals are the true leaders of men in times of war. Often veterans of years, if not decades of conflict, Generals are the people who best, but not always, understand how to fight and how to win. They are typically much rarer than Heroes but every army, save for the very poor or very desperate ought to have at least one.
\end{itemize}

\subsection{Unit Characteristics: The Profile}
\paragraph{•}
What follows below will be a list and explanation of the characteristics that make up units. Largely, the profiles of units will not change over the course of battle. This will allow for a fairly easy time to remember their specific statistics.

The full list of statistics will be present in the Army Building Section later in this document but the general individual statistics are discussed here.

\begin{itemize}
\item Type: This is the type of the unit. This will be important when making considerations for army building as some units will have more allowances than others.
\item Attacks:This is the number of attack dice that will be generated per stand in the unit of the listed type. There are typically two numbers with the first being the number of melee attacks generated and the second the number of ranged attacks generated. For example 1/2 indicates that each stand of the unit has 1 melee attack and 2 ranged attacks.
\item Defence: This is a number that indicates how resilient the unit, and all of its stands, are to damage. This number is used when determining casualties as wound rolls above this value will cause sufficiently serious damage.
\item Size:This is the size of the unit in terms of stands. There are typically two values. The first is the minimum number of stands for the unit when purchased and the second is the maximum number of stands that a unit may have from purchase.
\item Special:A placeholder for any particular special rules that a unit may have.
\end{itemize}

\subsubsection{Command Rating}
A Command Rating is a specific value that measures how capable a Commander unit will be when issuing Orders to units in his regiment. It is a value that will be specified that will be used in the proscribed manner for Command Checks.
\subsection{The Regiment}
\paragraph{}
Units do not operate by themselves typically. Rather, several units will be trained and deployed together so that they can cooperate with each other in order to provide necessary capacities that other units may lack. Pikemen and Musketeers are typically formed up together so that the Pikemen and interpose themselves in front of any dangers to the Musketeers and the Musketeers can fire upon distant enemy formations. Frequently, reserve infantry will be grouped with artillery formations to protect the artillery from enemy attack. Cavalrymen will typically operate by themselves but it is not unknown to see cavalry coordinate with infantry in order to deliver a series of decisive attacks.

\paragraph{}
The grouping of units on the battlefield is called a Regiment. The actual size and composition of a regiment may drastically differ across armies, nations or even time periods but for the purposes of gameplay, the Regiment defines a collection of units that operates under the command of a single Commander.

If the unit is the basic fighting component of a battle, the regiment is the basic organisational unit. When building armies, armies will be built in terms of regiments who will be assigned units. The Commander will command all of the units in his Regiment and no more.

\subsection{The Commander}
Individual units will have commanders, sergeants and officers who will individually direct the soldiers under their command. However, these leaders are too specific to the focus of the game. Rather,we focus on the more specific leaders, Commanders, who will direct the efforts of several units under their command. Similarly, a Commander unit, is not merely made up of a single person. Rather, it will be comprised of many people who all work to protect, support and advise the commander on the battlefield and inform their decision making processes, as well as keeping them alive.
\subsection{The Army}
\paragraph{•}
The Army is the fighting unit of players in terms of the game. Typically a game will be composed of one army per player and each army will oppose the other on the field of battle.

Armies are rather high level and have little additional points that directly affect the player and the game.
\subsubsection{The Role and Capacity of Commanders}
If units and regiments are the lifeblood of the army, then Commanders are the brains. They are vitally important to conducting the battle and have a number of special properties which will be outlined below.

\paragraph{Commanders and Damage:} Commanders are different in that they cannot attack like other units are able to. Also different is that Commanders cannot be directly damaged except in special circumstances which will be explained in their relevant sections.

\paragraph{Loss of Commanders:} Commanders can be killed in special circumstances. Due to the rarity of the events, this goes hand in hand with dire consequences for the regiment they command.

In the event that a Commander unit is killed, every unit that remains in his regiment must take a Morale Check, explained later on. If the check is failed, the unit is immediately removed from the table, having decided to flee the battle when their commander was lost. Once all Morale Checks have been taken, the surviving units will then "promote" a new Commander amongst them, typically the senior most surviving captain, to take command. This is reflected by the fact that you then choose a unit in your regiment to become the new Commander unit with all of the associated rules, losing any earlier rules that it had.
\\
Further note that the type, or quality, of the newly promoted Commander unit is always worse than or equal to the quality of the unit that was lost to reflect an unprepared or novice leader suddenly taking charge.

\paragraph{•}
For example, General Morse is killed in battle. After taking the appropriate actions, a new leader is promoted to replace him. This new leader, Lieutenant Cross, is of the Hero Rank whereas Morse was of the General rank. Later on in the battle, terrible misfortune strikes and Lieutenant Cross is killed as well. What little of the regiment that remains, elects Lieutenant Hobbs as the new Commander, of Hero rank as Cross was also of Hero rank who will hopefully lead them to safety.

\paragraph{•}
As as result of the severe consequences, Commanders should ideally be kept from the thickest peril save for the greatest need.
\section{Orders}
\paragraph{•}
Orders are the defining component of this game because they reflect a certain sophistication of military units that would ordinarily not be present in earlier periods. Units are trained to respond to commands specifically made by officers who lead them. While not a perfect system, and prone to mishap, the system allows for a great deal more to be done by soldiers in terms of battlefield operations and fighting.

\paragraph{•}
The orders listed here will be done in the Orders phase of the game. How to issue orders, respond to them and what each order does, and who it can be issued to, will be described below.

\subsection{Issuing Orders}
\subsubsection{Who can issue Orders?}
\paragraph{•}
Orders can only be issued by units with Novice,Hero or General type. These units are called Commanders regardless of rank. No other unit can issue orders.
\subsubsection{Who can receive Orders?}
\paragraph{•}
A Commander unit, General, Hero or Novice, can only issue Orders to units that are within his regiment as assigned during army creation. Units within a regiment may only receive Orders that specify that they may be issued to them, defined by type. Furthermore, only a unit that is not Disordered may receive orders with the exception being the Rally Order.
\subsubsection{How do I issue Orders?}
\paragraph{•}
The process of issuing orders is a straightforward one. 

\begin{enumerate}
\item During the Orders phase of your turn, select one of your Commanders that hasn't been selected already.
\item Select one of your units in his regiment that has not already received an order.
\item Decide which orders to give this unit.
\item Now indicate to your opponent what Orders this unit will be receiving, possibly using tokens placed next to the unit to permanently indicating this.
\item Take a Command Roll with your Commander. This is a roll taken by rolling 2d6 and comparing the result to the Command rating of the Commander in question. If your result is higher than the Commander's Command Rating, then the Command Roll has failed and your unit has either failed to receive its orders or not understood. Either way both that unit and your Commander may not receive and issue orders for that Orders Phase respectively. In the other case, your result is less than the Command rating of your Commander. For each point your result is under the Commander's Command rating, one order, in order of issue, is accepted by the unit. If there are not sufficient points for all of the orders on a unit, then the excess orders are discarded.
\item Repeat this process with your commander and all of the units in his regiment until no new units can receive Orders
\item Repeat the overall process for all regiments in your army until all Commanders have finished issuing their Orders.
\end{enumerate}

\paragraph{•}
Note that a Commander need not issue orders to all the units in his regiment. If a unit does not receive an order, it will default to the Stand Ready.

\paragraph{•}
Before continuing, it is worth discussing the potential implications of this system to the player. The theoretical maximum limit to how many orders a unit can perform in a turn is 8, based on a maximum Command rating of 10 and a minimum dice roll of 2. In practice, players will seldom roll so well with such a high Command rating. Rather, it is more likely that between 1 and 3 orders per unit will be doable per turn and for large regiments, even decent Commanders are likely to fail when forced to order many units around. With that in mind, it is imperative the player consider their orders, and the sequence in which they are issued, for units carefully as it is unwise to gamble on always getting sufficient points to perform complex or many actions.

\subsection{Regiment-wide Orders}
\paragraph{•}
For sufficiently large regiments, or even as a time saving measure, it is possible for a player to issue a single set of orders and have them be carried out by multiple, if not all, of the units in a regiment. This is called a Regimental Order and has a number of additional characteristics on top of conventional Orders. These differences are described below.

\begin{enumerate}
\item Before performing a Regimental Order, declare which of the units in the regiment are going to be involved in the order. 
\item Make sure that the units you have chosen are eligible to receive the Order(s) you are about to issue.
\item Declare your orders as per normal.
\item Take a single Command Roll as you normally would. However, there may be a penalty applied to your roll this time. If any of the conditions below are true, apply the specific penalty,once and only once, to the Command Roll by subtracting from your Commander's Command rating, the penalty value listed before comparing the result of the roll against the Command rating or otherwise perform the instruction as listed.
	\begin{itemize}
	\item If the number of units involved in the Regimental Order is less than or equal to the Commander's Command rating, the command penalty is 1.
	\item If the number of units involved in the Regimental Order is more than the Commander's Command rating, for each unit above, subtract 1 from the Commander's Command rating.
	\end{itemize}
\end{enumerate}

\paragraph{•}
Once a Regimental Order has been performed, then all of the units involved in the order will perform the order(s) as they are able. Note that a Commander can always still issue Orders after issuing a Regimental Order if units in his regiment not involved in the Regimental Order are left without Orders.
\subsection{Order List}
\subsubsection{Stand Ready}
\paragraph{•}
When a unit is issued the unit does not move but will be free to perform actions in later phases.
\subsubsection{March}
\paragraph{•}
When issued, a unit will be able to perform a single Full Advance from its current position, paying attention to specific movement rules as dictated by the rules of a Full Advance.
\subsubsection{Charge}
\paragraph{•}
A Charge is a specific form of movement action that brings two units into melee combat with one another. This is the only way that melee combat between two enemy units can happen. A Charge Order requires a target for the ordered unit and only ever one Charge Order may be given to a unit regardless of how many orders it may perform.

See the rules in the Movement section for details on how Charges function once ordered.
\subsubsection{Guard}
\paragraph{•}
This is a special action whereby a unit will enter a protective stance of a designated target unit and then act against enemy threats to the protected unit. To perform this order, you must designate a unit in the same regiment as the unit receiving the order as well as within 4 inches of the unit receiving the order. If the order goes off, the unit receiving the order does not move, remaining in their position but they gain 1 Guard Token.

\paragraph{•}
Guard tokens, placed near the unit for clarity, can only be used in enemy Order Phases. If an enemy unit declares a Charge order with one of his units against a unit that has a unit guarding it with a Guard Token, then the following will happen:

\begin{enumerate}
\item The unit that has the Guard Token rolls a d6. 
\item If the result is a 1, the unit has blundered its movement and the charging enemy unit completes its Charge as per normal. The unit with the token also cannot act until the next turn. 
\item If the result is greater than 1,then the unit has successfully spurred into action. The charging enemy unit must now redirect their charge towards the unit with the Guard Token. 
\begin{enumerate}
\item If they are unable to complete their charge, they will move half the distance towards the unit with the Token. 
\item If they are, then they must complete their charge against the unit with the Guard Token. 

\end{enumerate}
\item Regardless, it then loses its Guard Token.
\item Note that the unit with the Guard Token must have line-of-sight to the enemy that is attacking the unit that it is protecting.
\end{enumerate}

\paragraph{•}
Only 1 token of any kind may be active on a unit at any time. If a unit gains a token while already holding a token, it will replace its token for the new one.
\subsubsection{Support}
\paragraph{•}
This is a special kind of action whereby one unit will then support another unit when attacking an enemy unit. The unit with the Support Order will designate one unit in its regiment to Support. This is called the Supported Unit. When the Supported Unit declares a Charge against an enemy unit, or is subject to a Charge by an enemy unit, the unit with the Support Order will automatically declare a charge against the target of the Supported Unit or the unit charging it. Note that this is still subject to the normal rules of Charges.

\subsubsection{Change Formation}
\paragraph{•}
When a unit receives this order, the owning player may change its formation as he wishes provided the final formation is still valid according to the rules of Formations. 
\subsubsection{About Face}
\paragraph{•}
The unit immediately rotates in place up to 180 degrees. This counts as having moved for Shooting purposes.
\subsubsection{Rally}
\paragraph{•}
When a unit receives a Rally order, it takes a Command Roll. If the Command Roll is passed, the unit is no longer Disordered. If the roll is failed, the unit is still Disordered and may not receive any other orders until it is no longer Disordered.

\section{Movement}
\subsection{March}
\paragraph{•}
A unit that has received a March order will move. It may move a total of 6 inches in total. During the movement, it may move in the following ways
\begin{itemize}
\item Move straight forward
\item Pivot up to 45 degrees in either direction
\item Move backwards
\item Move directly sideways, left or right.
\end{itemize}
\subsection{Charge}
\paragraph{•}
A Charge is kind of Movement order that brings enemy units into close quarters combat that will result in melee. There are a number of components to a charge.

\begin{itemize}
\item The first requirement for a charge is a target. The target enemy unit must be in the front arc of a unit that you want to charge with and in line-of-sight.
\item The unit charging will charge a distance of d6+2 inches.
\item If the unit is able to make base contact with the target unit, then the charge is successful.
\item If the unit is unable to charge, it then moves forward the distance indicated by the die roll.
\end{itemize}

\paragraph{•}
Be wary of charges as they are not as predictable as conventional movement and estimating poorly will result in a unit being drawn too close an enemy unit and possibly subject to enemy unit charges later on.

\section{Ranged Combat}
\paragraph{•}
Ranged Combat 
\subsection{Artillery}
\subsection{Missile Troops}
\section{Close Combat}
\section{Morale and your Soldiers}
\section{The Game Turn}
\subsection{Phases of the Game and Player Turns}
\section{Army Building}
\end{document}
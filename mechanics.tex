\documentclass[12pt]{article}
\begin{document}
\begin{titlepage}
\begin{huge}
\begin{center}
\title{Game Mechanics}
Game Mechanics
\end{center}
\end{huge}
\end{titlepage}
\pagebreak
\section{Game Mechanics}
\subsection{Army Organisation}
\paragraph{•}
Armies are organised along two lines: the formation and the unit. Formations are the basic logical and organisational unit of armies and consist of at least a Commander and a Unit although in most cases, a Formation will be more than just one unit. The basic combat component of armies are units. Units represent groups of people or entities that fight together usually with a local leader under the command of a larger authority: the Commander.
\subsubsection{Units}
\paragraph{}A unit will be how you the player fights your opponent. The unit can be destroyed in combat or forced to flee. The unit has a number of capacities. These capacities are largely expanded upon in the unit types section. Units will have a Defensive Value which is their defence score from 3 to - with - indicating no defence value. Furthermore some units have a Contest Score which indicates how good they are in prolonged battles against other units in terms of how many dice they have to fight contests. 
\subsubsection{Formations}
\paragraph{}
Formations are made up of multiple units. The size of the formation is up to you as well as its composition but one common rule binds them all: Coherence. Coherence means that all the units in the formation are within a commanding distance to their commander. What this means is that all of the units are within some distance to their commander, determined by their command stat, that indicates the range at which a commander can command units in his formation.

\paragraph{•}
The Coherence distance is measured from the Commander of the formation to any unit in their formation. A unit must remain at most, 2 times the command stat. This is a straight line distance and does not require a line-of-sight from commander to unit; we assume that the commander has a capacity to maintain communications over distance to his unit at some distance but further, this breaks down. If a unit is further from their commander, it is no longer in Coherence with the rest of its formation. Only units in Coherence can benefit from Activation. A unit out of Coherence can only move back towards Coherence. The exception to this would be units contested by other units and moved out of Coherence. These units are locked into Contest but must return as soon as they can if they are free in later turns. Troops in Contest are presumably too occupied by their fighting to bother with the battle as a large.

\subsubsection{Commanders}
\paragraph{•}
Commanders are a special type of unit in the army. They are invulnerable on the field to enemy attacks and cannot be killed by the enemy. In this way, we think of them as tokens rather than actual fighting forces. They have only one statistic: Command. This is their capacity to command their troops. It will be a number from about 6 to 9 with 9 being a great commander and 6 being only a barely qualified leader.
\paragraph{•}
A commander can move up to 40cm in a single turn, which happens only after all the units in his formation have taken their actions.

\subsection{Unit Types}
\paragraph{•}
All units occupt a space of 40mm by 40mm this refers to their base size and any number of appropriate models can be based on them.
\subsubsection{Infantry}
\paragraph{•}
Infantry units refer to units that fight on foot. There is a wide variety of possibilities to encapsulate infantry units but they on principle, operate in the same generic way.

\begin{itemize}
\item Infantry move up to 20cm in a single advance.
\item Infantry can make Infantry Assaults, Advances, Contests,Barrages(if armed as such).
\item Infantry can only be in a contest with 1 enemy Infantry unit.
\item Up to 3 Infantry units can assault one enemy unit at a time.
\end{itemize}

\subsubsection{Cavalry}
\paragraph{•}
Cavalry units are soldiers who fight from some kind of platform creature. In this particular case, the creature is ridden by the soldier and grants them increased mobility, protection and shock force during a charge.
\begin{itemize}
\item Cavalry move up to 35cm in a single advance.
\item Cavalry can make Cavalry Assaults, Advances,Barrages(if armed as such).
\item Up to 2 Cavalry units can assault one enemy unit at a time.
\end{itemize}

\subsubsection{Artillery}
\paragraph{•}
Artillery units are machines or mechanisms to project force and firepower across a great distance. Some Artillery is useful only against attacking fortifications. Others are designed to cause damage to massive enemy formations.

\begin{itemize}
\item Artillery move up to 10cm in a single advance.
\item Artillery can make Advances and Barrages.
\item Artillery units are automatically destroyed if Assaulted or Contested by enemy Infantry or Cavalry units.
\end{itemize}

\subsubsection{Special}
\paragraph{•}
Special Units are specialised units that do not fit into a traditional hierarchy of soldier organisation. They are either so unique by nature or fulfil so unique a role, that they do not fit into any other scheme. Largely these units will be defined by their special rules above all else.

\begin{itemize}
\item Special Units move up to 25cm in a single advance.
\item Special Units can make Assaults, Advances, Contests,Barrages(if armed as such).
\item Special Units  can only be in a contest with 1 enemy Infantry unit.
\item Up to 1 Special Unit can assault one enemy unit at a time.
\end{itemize}

\subsection{Turn Mechanics}
\paragraph{•}
The game is made up of a number of turns. A consists of the activation of one formation by player. A formation activation is when at the start of the turn, you indicate to your opponents that one of your formations will be activated. You will then take the activation roll. This is rolling 2d6 dice and comparing the result to your formation's commander's command stat. If the result is less than or equal to this result, the formation is successfully activated. Furthermore, compare the activation results between players. The order of play is determined by the lowest result rolled. If two players are activating their formations and both pass, the player with the lowest activation roll will move and act with his formation first. In the event of ties, the smallest formation will go first. If there are further ties, both players will roll a d6 dice until one player rolls lower than the other.

\subsubsection{Activation of Formations}
\paragraph{•}
If a player has failed their command roll, then their formation will not be as effective. This represents a great deal of confusion on the part of commander and units with regards to what to do as both sides try to communicate when a breakdown of communications has occurred. In this case,all units only get 1 Action. Should the activation be successful however, then all units in the formation get 2 Actions.

\paragraph{•}
Once a unit has been activated, and the order of formations determined, it is time for units to take Actions. The priority scheme is as follows:
\begin{enumerate}
\item All units out of Coherence must move back into Coherence.
\item Choose a unit that has not taken any actions yet to take actions
\item Move your Commander 

\paragraph{•}
When taking actions with a unit, a unit must take all of its actions before another unit takes actions. If a unit can take more than 1 action, you may choose to ignore further actions after taking one action. All actions are resolved before moving onto new units.

\paragraph{•}
After all units have activated, the formation is finished and the enemy formation will take its turn.
\end{enumerate}
\subsubsection{Advances}
\paragraph{•}
Advances consist of units moving across the battlefield attempting to maintain a fighting stance while doing so. A single advance means that a unit can move as far as its type allows in a direction of its choosing. You are allowed to change the direction of the facing of the unit once per advance by up to 90 degrees. Advances can bring the unit into base contact with the enemy, starting Contests.

\subsubsection{Assaults}
\paragraph{Infantry Assaults}
These are assaults conducted by infantry to rapidly attack and damage enemy units in a formation. A unit must select a target enemy unit within 10cm of it and then roll a d6. If the result is 5+, then the assault is successful. Roll a d6 and compare the result to the enemy Defence value. If the result is higher, the enemy unit is destroyed by the assault and the assaulting unit moves forward up to 10cm. If the assault fails, then the assaulting unit remains where it was as the enemy unit was able to fight them off and force them to retreat. Assaults require line-of-sight to the target.
\paragraph{Cavalry Assaults}
These are assaults conducted by Cavalry to rapidly attack and damage enemy units in a formation and then flee before they can be bogged down in a mire. A unit must select a target enemy unit within 20cm of it and then roll a d6. If the result is 4+, then the assault is successful. Roll a d6 and compare the result to the enemy Defence value. If the result is higher, the enemy unit is destroyed by the assault and the assaulting unit remains where it was. We assume that the cavalry unit charged and reformed back at their original position. If the result is less than this however,the cavalry unit has been bogged down in the charge. Now, the cavalry unit is placed in base contact with the enemy unit it failed to flee from. This means our Cavalry unit has become contested.

\subsubsection{Contests}
\paragraph{•}
A contest consists of two units that are fighting in base-to-base contact with each other. Unlike Assaults which end quickly and are generally not likely to defeat units, Contests are means to both secure positions and destroy units.

\paragraph{•}
Once two units have made base contact a Contest begins. Once in a Contest, a unit cannot do anything other than Fight the Contest or Flee. You decide at the activation of your formation what you wish to do with units currently in contests.

\paragraph{•}
If you choose to Flee,take a Command check with your commander at -2 to their command stat. If you are successful, make an advance with your unit in the contest away from their opponent. This cannot be used to bring them into a new Contest with other enemy units. If the test is failed, your unit takes a Free Hit by the enemy using the process described for assaults. Roll a d6, based on the type of units involved, and proceed from there. This can destroy your unit. However, because it is easier to assault a fleeing enemy, reduce the successful roll score needed by 1. Cavalry succeed on a 3+ and infantry on a 4+ therefore.

\paragraph{•}
If you choose to fight the Contest, then roll all of your contest dice for that unit. Your enemy will do the same. Now match the highest pairs of dice to each other. The winner of the Contest is the one with the most number of win pairs. They move their enemy unit,and follow still in base contact, a number of cm equal to the total of their Contest dice rolled. If this amount is greater than half their enemy's standard movement, that enemy unit is destroyed and your unit is free to continue to move the amount.
\subsubsection{Shooting}
\paragraph{•}
There are two kinds of shooting: Artillery and Infantry/Cavalry shooting. We describe both.

\paragraph{Artillery}
When artillery shoots, it is not to target individual units, rather, Artillery shoots masses of material to force enemy formations to halt and disrupt themselves. When shooting with artillery, rather than select an enemy unit, select an enemy formation within the range specified of the weapons. Roll a d6 for each unit in the enemy formation that is not currently Contesting an enemy unit. We consider these units either shielded from artillery by way of their melee or they are too focused by the battle to consider the artillery raining on them. For each value of 4-5, add one Panic Token to the Formation. For each 6, destroy one enemy unit in the formation chosen by their owner. After the rolls, the enemy formation must take a command check subtracting from his Command rating, each panic token. If he passes the check, nothing happens. However, should he fail, all units not Contesting must move half an advance in a direction away from the source of the Artillery. 

\paragraph{Non-Artillery Shooting}
This is when infantry or more rarely cavalry units armed with hand ranged weapons attack enemy units. This has a minor chance to do serious damage to units in comparison to artillery in general. All ranged attacks, called Barrages, unless otherwise stated have a range of 40cm.
\paragraph{•}
Select an enemy unit within 40cm that your unit can see. Roll a d6. On a 5+, you have hit the enemy unit with enough firepower. Now roll another d6 and compare it with the Defence value of the enemy. If your roll is higher, you have destroyed the enemy unit. Otherwise nothing happens. There are modifiers listed below.
\begin{itemize}
\item Add 1 to the opponent's Defence value if your unit has Advanced before shooting.
\item Add 1 to the opponent's Defence value if your target is more than 30cm away.
\item Add 1 to the opponent's Defence value if your target is in a Fortification
\end{itemize}
These modifiers stack with each other.
\end{document}